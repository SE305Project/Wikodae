%
% File acl2016.tex
%
%% Based on the style files for ACL-2015, with some improvements
%%  taken from the NAACL-2016 style
%% Based on the style files for ACL-2014, which were, in turn,
%% Based on the style files for ACL-2013, which were, in turn,
%% Based on the style files for ACL-2012, which were, in turn,
%% based on the style files for ACL-2011, which were, in turn,
%% based on the style files for ACL-2010, which were, in turn,
%% based on the style files for ACL-IJCNLP-2009, which were, in turn,
%% based on the style files for EACL-2009 and IJCNLP-2008...

%% Based on the style files for EACL 2006 by
%%e.agirre@ehu.es or Sergi.Balari@uab.es
%% and that of ACL 08 by Joakim Nivre and Noah Smith

\documentclass[12pt]{article}
\usepackage{acl2016}
\usepackage{times}
\usepackage{url}
% \usepackage{ctex}
\usepackage{amsmath,amscd,amsbsy,amssymb,latexsym,url,bm,amsthm}
\usepackage{amsmath}
\usepackage{amssymb}
\usepackage{graphicx}
\usepackage{hyperref}
\usepackage{listings}
\lstset{
basicstyle=\footnotesize\ttfamily,
breaklines=true,
}

\usepackage{listings}
\usepackage{color}

\aclfinalcopy

% \definecolor{mygreen}{rgb}{0,0.6,0}
% \definecolor{mygray}{rgb}{0.5,0.5,0.5}
% smaller than 5cm (the original size); we will check this
% in the camera-ready version and ask you to change it back.

\newcommand\BibTeX{B{\sc ib}\TeX}

\title{SE305 COURSE PROJECT FINAL REPORT}

\author{Jeremy Liu, Qianyang Peng, Jingyu Cui}
\date{11/28/2016}
% Useful Command:
% \url{acl2016.org/index.php?article_id=9}
% {\small\verb|***|}
% {\small\tt acl2016.pdf}
% wew\ref{sec:abs}

% \begin{quote}
% \begin{verbatim}
% \usepackage{times}
% \usepackage{latexsym}
% \end{verbatim}
% \end{quote}

% \begin{table}[h]
% \begin{center}
% \begin{tabular}{|l|rl|}
% \hline \bf Type of Text & \bf Font Size & \bf Style \\ \hline
% paper title & 15 pt & bold \\
% author names & 12 pt & bold \\
% \hline
% \end{tabular}
% \end{center}
% \caption{\label{font-table} Font guide. }
% \end{table}

% footnote here\footnote{This is how a footnote should appear.}


% \begin{table}
% \centering
% \small
% \begin{tabular}{cc}
% \begin{tabular}{|l|l|}
% \hline
% {\bf Command} & {\bf Output}\\\hline
% \verb|{\"a}| & {\"a} \\
% \verb|{\^e}| & {\^e} \\
% \verb|{\aa}| & {\aa}  \\\hline
% \end{tabular} &
% \begin{tabular}{|l|l|}
% \hline
% {\bf Command} & {\bf  Output}\\\hline
% \verb|{\c c}| & {\c c} \\
% \verb|{\l}| & {\l} \\
% \verb|{\ss}| & {\ss} \\\hline
% \end{tabular}
% \end{tabular}
% \caption{Example commands for accented characters, to be used in, e.g., \BibTeX\ names.}\label{tab:accents}
% \end{table}

% \label{sec:intro}
% wew\ref{sec:intro}

% \begin{figure}[!htp]
% \centering
% \includegraphics[width=0.5\linewidth]{Top20.png}
% \caption{Top-20 words from class "appearance"}
% \end{figure}


% \begin{quote}
% ``Gusfield \shortcite{Gusfield:97,Aho:72} \cite{APA:83}   showed that ...''
% \end{quote}
\begin{document}
\maketitle
\tableofcontents
\newpage
\pagenumbering{arabic}

\begin{abstract}
	With the developing ofacknowledgement.
\end{abstract}

\section{ER Models}
Our system is 

\section{Table Designs}
We finished out table design according to our ER model. Our final design contains 9 tables. Reference information is not stored in our tables, while all other information are stored. The design of our tables is elaborated below:
\subsection{entity}
\lstset{language=SQL}
Table schema:
\begin{lstlisting}
CREATE TABLE IF NOT EXISTS `wikidata`.`entity` (
  `serial_id` BIGINT(32) NOT NULL AUTO_INCREMENT,
  `entity_id` VARCHAR(32) NOT NULL,
  `entity_language` VARCHAR(16) NOT NULL,
  `entity_type` VARCHAR(16) NULL DEFAULT NULL,
  `entity_text` VARBINARY(255) NULL DEFAULT NULL,
  PRIMARY KEY (`serial_id`),
  INDEX `EID` (`entity_id` ASC),
  INDEX `ELANG` (`entity_language` ASC),
  INDEX `ETYPE` (`entity_type` ASC))
ENGINE = InnoDB
\end{lstlisting}
The data strored in corresponding fields:
\begin{itemize}
\item \textbf{serial\_id}: Auto incremental serial ID. As language information is introduced here and entity ID can not be used as Primary Key, so we use this serial\_id here as a Primary Key.
\item \textbf{entity\_id}: ID of the corresponding entity. Eg. Q5, P110, Q123423, etc.
\item \textbf{entity\_language}: Language of the label of the corresponding entity. Eg. Zh-cn, en, etc.
\item \textbf{entity\_type}: The entity type identifier. "item" for data items, and "property" for properties.
\item \textbf{entity\_text}: Contains the labels in different languages. Eg. square kilometre, Kilometro quadrato, Vierkante kilometer, etc.
\end{itemize}
Indexed tables: besides indexing the Primary Key serial\_id, other keys including entity\_id, entity\_language and entity\_type are also indexed.
\subsection{description}
\lstset{language=SQL}
Table schema:
\begin{lstlisting}
CREATE TABLE IF NOT EXISTS `wikidata`.`description` (
  `serial_id` BIGINT(32) NOT NULL AUTO_INCREMENT,
  `entity_id` VARCHAR(32) NOT NULL,
  `desc_language` VARCHAR(8) NULL,
  `desc_text` VARBINARY(255) NULL,
  PRIMARY KEY (`serial_id`),
  INDEX `EID` (`entity_id` ASC),
  INDEX `DLANG` (`desc_language` ASC))
ENGINE = InnoDB
\end{lstlisting}
The data strored in corresponding fields:
\begin{itemize}
\item \textbf{serial\_id}: Same as in table 'entity'.
\item \textbf{entity\_id}: Same as in table 'entity'.
\item \textbf{desc\_language}: Similar to entity\_language in table 'entity'.
\item \textbf{desc\_text}: Similar to entity\_text in table 'entity'.
\end{itemize}
Indexed tables: besides indexing the Primary Key serial\_id, other keys including entity\_id, desc\_language and are also indexed.
\subsection{mainsnak}
\lstset{language=SQL}
Table schema:
\begin{lstlisting}
CREATE TABLE IF NOT EXISTS `wikidata`.`mainsnak` (
  `snak_id` VARCHAR(64) NOT NULL,
  `entity_id` VARCHAR(32) NOT NULL,
  `property_id` VARCHAR(32) NOT NULL,
  `serial` INT(4) NOT NULL,
  `claimtype` VARCHAR(32) NULL DEFAULT NULL,
  `snaktype` VARCHAR(32) NULL DEFAULT NULL,
  `datatype` VARCHAR(32) NULL DEFAULT NULL,
  `rank` VARCHAR(32) NULL DEFAULT NULL,
  PRIMARY KEY (`snak_id`),
  INDEX `EID` (`entity_id` ASC),
  INDEX `PID` (`property_id` ASC),
  INDEX `CTYPE` (`claimtype` ASC),
  INDEX `STYPE` (`snaktype` ASC),
  INDEX `DTYPE` (`datatype` ASC))
ENGINE = InnoDB
\end{lstlisting}
The data strored in corresponding fields:
\begin{itemize}
\item \textbf{snak\_id}: An arbitrary identifier for the claim, used as Primary Key of this table.
\item \textbf{entity\_id}: ID of the entity this claim belongs to.
\item \textbf{property\_id}: The property this claim is describing.
\item \textbf{serial}: This field is X means this is the Xth claim of this property.
\item \textbf{claimtype}: the type of the claim - currently either statement or claim.
\item \textbf{snaktype}: The type of the snak. Currently, this is one of value, somevalue or novalue.
\item \textbf{datatype}: The datatype field indicates how the value of the Snak can be interpreted. This fields indicate the concrete table begining with "datavalue\_"
\item \textbf{rank}: The rank expresses whether this value will be used in queries, and shown be visible per default on a client system. The value is either preferred, normal or deprecated.
\end{itemize}
\subsection{datavalue\_string}
\lstset{language=SQL}
Table schema:
\begin{lstlisting}
CREATE TABLE IF NOT EXISTS `wikidata`.`datavalue_string` (
  `snak_id` VARCHAR(64) NOT NULL,
  `value` VARBINARY(255) NULL DEFAULT NULL,
  PRIMARY KEY (`snak_id`))
ENGINE = InnoDB
\end{lstlisting}
The data strored in corresponding fields:
\begin{itemize}
\item \textbf{snak\_id}: Primary key. Refering to the table mainsnak.
\item \textbf{value}: The content of the string.
\end{itemize}
\subsection{datavalue\_time}
\lstset{language=SQL}
Table schema:
\begin{lstlisting}
CREATE TABLE IF NOT EXISTS `wikidata`.`datavalue_time` (
  `snak_id` VARCHAR(64) NOT NULL,
  `time` VARCHAR(64) NULL DEFAULT NULL,
  `timezone` VARCHAR(32) NULL DEFAULT NULL,
  `before` VARCHAR(32) NULL DEFAULT NULL,
  `after` VARCHAR(32) NULL DEFAULT NULL,
  `precision` INT(8) NULL DEFAULT NULL,
  `calendarmodel` VARCHAR(255) NULL DEFAULT NULL,
  PRIMARY KEY (`snak_id`))
ENGINE = InnoDB
\end{lstlisting}
The data strored in corresponding fields:
\begin{itemize}
\item \textbf{snak\_id}: Primary key. Refering to the table mainsnak.
\item \textbf{time}: The content of the string.
\item \textbf{timezone}: Signed integer. Now unused.
\item \textbf{before}: Begin of an uncertainty range, given in the unit defined by the precision field. This cannot be used to represent a duration.
\item \textbf{after}: End of an uncertainty range, given in the unit defined by the precision field. This cannot be used to represent a duration.
\item \textbf{precision}: To what unit is the given date/time significant.
\item \textbf{calendarmodel}: A URI of a calendar model, such as gregorian or julian. Typically given as the URI of a data item on the repository.
\end{itemize}
\subsection{datavalue\_globecoordinate}
\lstset{language=SQL}
Table schema:
\begin{lstlisting}
CREATE TABLE IF NOT EXISTS `wikidata`.`datavalue_globecoordinate` (
  `snak_id` VARCHAR(64) NOT NULL,
  `latitude` FLOAT NULL DEFAULT NULL,
  `longitude` FLOAT NULL DEFAULT NULL,
  `altitude` FLOAT NULL DEFAULT NULL,
  `precision` FLOAT NULL DEFAULT NULL,
  `globe` VARCHAR(255) NULL DEFAULT NULL,
  PRIMARY KEY (`snak_id`))
ENGINE = InnoDB
\end{lstlisting}
The data strored in corresponding fields:
\begin{itemize}
\item \textbf{snak\_id}: Primary key. Refering to the table mainsnak.
\item \textbf{latitude}: The latitude part of the coordinate in degrees, as a float literal (or an equivalent string).
\item \textbf{longitude}: The longitude part of the coordinate in degrees, as a float literal (or an equivalent string).
\item \textbf{altitude}: Deprecated and no longer used. Will be dropped in the future.
\item \textbf{precision}: the coordinate's precision, in (fractions of) degrees, given as a float literal (or an equivalent string).
\item \textbf{precision}: To what unit is the given date/time significant.
\item \textbf{globe}: the URI of a reference globe.
\end{itemize}
\subsection{datavalue\_quantity}
\lstset{language=SQL}
Table schema:
\begin{lstlisting}
CREATE TABLE IF NOT EXISTS `wikidata`.`datavalue_quantity` (
  `snak_id` VARCHAR(64) NOT NULL,
  `amount` VARCHAR(64) NULL DEFAULT NULL,
  `upperBound` VARCHAR(64) NULL DEFAULT NULL,
  `lowerBound` VARCHAR(64) NULL DEFAULT NULL,
  `unit` VARCHAR(64) NULL DEFAULT NULL,
  PRIMARY KEY (`snak_id`))
ENGINE = InnoDB
\end{lstlisting}
The data strored in corresponding fields:
\begin{itemize}
\item \textbf{snak\_id}: Primary key. Refering to the table mainsnak.
\item \textbf{amount}: The nominal value of the quantity, as an arbitrary precision decimal string. The string always starts with a character indicating the sign of the value, either "+" or "-".
\item \textbf{upperBound}: Optionally, the upper bound of the quantity's uncertainty interval, using the same notation as the amount field.
\item \textbf{lowerBound}: Optionally, the lower bound of the quantity's uncertainty interval, using the same notation as the amount field.
\item \textbf{unit}: the URI of a unit (or "1" to indicate a unit-less quantity).
\end{itemize}
\subsection{datavalue\_wikibase}
\lstset{language=SQL}
Table schema:
\begin{lstlisting}
CREATE TABLE IF NOT EXISTS `wikidata`.`datavalue_wikibase` (
  `snak_id` VARCHAR(64) NOT NULL,
  `id` VARCHAR(32) NULL DEFAULT NULL,
  PRIMARY KEY (`snak_id`))
ENGINE = InnoDB
\end{lstlisting}
The data strored in corresponding fields:
\begin{itemize}
\item \textbf{snak\_id}: Primary key. Refering to the table mainsnak.
\item \textbf{id}: defines the id of the entity.
\end{itemize}
\subsection{qualifier}
\lstset{language=SQL}
Table schema:
\begin{lstlisting}
CREATE TABLE IF NOT EXISTS `wikidata`.`qualifier` (
  `serial_id` BIGINT(32) NOT NULL AUTO_INCREMENT,
  `hash` VARCHAR(64) NULL DEFAULT NULL,
  `snaktype` VARCHAR(32) NULL DEFAULT NULL,
  `property_id` VARCHAR(32) NULL DEFAULT NULL,
  `datatype` VARCHAR(32) NULL DEFAULT NULL,
  INDEX `STYPE` (`snaktype` ASC),
  INDEX `PID` (`property_id` ASC),
  INDEX `DTYPE` (`datatype` ASC),
  PRIMARY KEY (`serial_id`))
ENGINE = InnoDB
\end{lstlisting}
The data strored in corresponding fields:
\begin{itemize}
\item \textbf{serial\_id}: Primary key.
\item \textbf{hash}: Hash value, may be duplicated.
\item \textbf{snaktype}: Similar to the table mainsnak.
\item \textbf{property\_id}: Similar to the table mainsnak.
\item \textbf{datatype}: Similar to the table mainsnak.
\end{itemize}
\section{}

This project i

\section{ACKNOWLEDGEMENT}

Our projec
\end{document} 